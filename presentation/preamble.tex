% !TEX root = ../presentation.tex
% Presentation Preamble

%%%%%%%%%%%%%%%%%%%%%%%%%%%%%%%%%%%%%%%%%%%%%%%%%%%%%%%%%%%%%%%%%%%%%%%%%%
%   BEAMER CLASS
%%%%%%%%%%%%%%%%%%%%%%%%%%%%%%%%%%%%%%%%%%%%%%%%%%%%%%%%%%%%%%%%%%%%%%%%%%

\documentclass[xcolor={dvipsnames}]{beamer}

\usetheme{default}
\usecolortheme[named=orange]{structure}
\setbeamertemplate{sections/subsections in toc}[sections numbered]
\setbeamertemplate{items}[default]
\usefonttheme[onlymath]{serif}
\setbeamertemplate{caption}{\raggedright\insertcaption\par}

%%%%%%%%%%%%%%%%%%%%%%%%%%%%%%%%%%%%%%%%%%%%%%%%%%%%%%%%%%%%%%%%%%%%%%%%%%
% 	GENERAL PACKAGES
%%%%%%%%%%%%%%%%%%%%%%%%%%%%%%%%%%%%%%%%%%%%%%%%%%%%%%%%%%%%%%%%%%%%%%%%%%

\usepackage[american]{babel}
\usepackage{hyperref}
\usepackage{graphicx}
\usepackage{xcolor}
\usepackage{textcomp}

%%%%%%%%%%%%%%%%%%%%%%%%%%%%%%%%%%%%%%%%%%%%%%%%%%%%%%%%%%%%%%%%%%%%%%%%%%
% 	MATH
%%%%%%%%%%%%%%%%%%%%%%%%%%%%%%%%%%%%%%%%%%%%%%%%%%%%%%%%%%%%%%%%%%%%%%%%%%

\usepackage{amsmath}
\usepackage{blkarray}
\usepackage{gensymb}
\usepackage{relsize}
\usepackage{cancel}

%%%%%%%%%%%%%%%%%%%%%%%%%%%%%%%%%%%%%%%%%%%%%%%%%%%%%%%%%%%%%%%%%%%%%%%%%%
% 	DRAWING
%%%%%%%%%%%%%%%%%%%%%%%%%%%%%%%%%%%%%%%%%%%%%%%%%%%%%%%%%%%%%%%%%%%%%%%%%%

\usepackage{tikz}
\usepackage{pgfplots}

\pgfplotsset{compat=1.3}
\usetikzlibrary{3d}
\usetikzlibrary{shapes}
\usetikzlibrary{3d}
\usepgflibrary{fpu}

%%%%%%%%%%%%%%%%%%%%%%%%%%%%%%%%%%%%%%%%%%%%%%%%%%%%%%%%%%%%%%%%%%%%%%%%%%
% 	CODE
%%%%%%%%%%%%%%%%%%%%%%%%%%%%%%%%%%%%%%%%%%%%%%%%%%%%%%%%%%%%%%%%%%%%%%%%%%

\usepackage{algorithm}
\usepackage[noend]{algpseudocode}
\usepackage{listings}

\definecolor{stringgreen}{RGB}{50,220,15}

\lstset{%
	basicstyle=\ttfamily\footnotesize,
	language=Python,
	aboveskip=3mm,
	belowskip=3mm,
	showstringspaces=false,
	columns=flexible,
	numbers=none,
	numberstyle=\tiny\color{red},
	keywordstyle=\color{magenta},
	commentstyle=\color{gray},
	stringstyle=\color{stringgreen},
	breaklines=true,
	breakatwhitespace=true,
	tabsize=2
}

%%%%%%%%%%%%%%%%%%%%%%%%%%%%%%%%%%%%%%%%%%%%%%%%%%%%%%%%%%%%%%%%%%%%%%%%%%
% 	CONFIG
%%%%%%%%%%%%%%%%%%%%%%%%%%%%%%%%%%%%%%%%%%%%%%%%%%%%%%%%%%%%%%%%%%%%%%%%%%

\graphicspath{{figures/}}
% \addtolength{\parskip}{\baselineskip}

%%%%%%%%%%%%%%%%%%%%%%%%%%%%%%%%%%%%%%%%%%%%%%%%%%%%%%%%%%%%%%%%%%%%%%%%%%
% 	BEAMER FOOTER
%%%%%%%%%%%%%%%%%%%%%%%%%%%%%%%%%%%%%%%%%%%%%%%%%%%%%%%%%%%%%%%%%%%%%%%%%%

\setbeamertemplate{footline}[text line]{%
  \parbox{\linewidth}{%
    \vspace*{-15pt}%
    \scriptsize
    Peter Goldsborough\hspace{1.3cm}%
    My Learning is Deeper Than Yours\hfill%
    \insertpagenumber}%
}
\setbeamertemplate{navigation symbols}{}

%%%%%%%%%%%%%%%%%%%%%%%%%%%%%%%%%%%%%%%%%%%%%%%%%%%%%%%%%%%%%%%%%%%%%%%%%%
% 	NEW COMMANDS
%%%%%%%%%%%%%%%%%%%%%%%%%%%%%%%%%%%%%%%%%%%%%%%%%%%%%%%%%%%%%%%%%%%%%%%%%%

\newcommand{\red}{\color{red}}

\newcommand{\inlineitem}[2]{{\color{orange}#1.} #2 \hspace{0.5cm}}

\newcommand{\textframe}[1]{%
  \begin{frame}
    \centering
    \Huge
    \color{orange}
    \vspace{0.2cm}
    #1
  \end{frame}
}

\newcommand\invisiblesection[1]{%
  \refstepcounter{section}%
  \addcontentsline{toc}{section}{\protect\numberline{\thesection}#1}%
  \sectionmark{#1}
}

\newcommand{\pitem}{\pause\item}

\DeclareSymbolFont{extraup}{U}{zavm}{m}{n}
\DeclareMathSymbol{\varheart}{\mathalpha}{extraup}{86}

\def\bitcoin{%
  \leavevmode
  \vtop{\offinterlineskip %\bfseries
    \setbox0=\hbox{B}%
    \setbox2=\hbox to\wd0{\hfil\hskip-.03em
    \vrule height .3ex width .15ex\hskip .08em
    \vrule height .3ex width .15ex\hfil}
    \vbox{\copy2\box0}\box2}}

\DeclareMathOperator*{\argmin}{arg\,min}
\DeclareMathOperator*{\softmax}{softmax}
\DeclareMathOperator*{\relu}{relu}

\newcommand{\frameheader}[1]{{\Large #1}\vspace{0.5cm}}

% x offset, y offset, z offset, length
\newcommand{\cube}[4]{
  \xyplane{#3} {
    \draw (#1, #2) rectangle ++(#4, #4);
  }
  \xyplane{#3 + #4} {
    \draw (#1, #2) rectangle ++(#4, #4);
  }
  \yzplane{#1}{
    \draw (#2, #3) rectangle ++(#4, #4);
  }
  \yzplane{{#1 + #4}}{
    \draw (#2, #3) rectangle ++(#4, #4);
  }
}

\newcommand{\xyplane}[2]{
  \begin{scope}[canvas is xy plane at z={#1}]
    #2
  \end{scope}
}

\newcommand{\xzplane}[2]{
  \begin{scope}[canvas is xz plane at y={#1}]
    #2
  \end{scope}
}

\newcommand{\yzplane}[2]{
  \begin{scope}[canvas is yz plane at x={#1}]
    #2
  \end{scope}
}

\newcommand{\random}[1]{\pdfuniformdeviate #1}

\newcommand{\randomcolor}{%
  \definecolor{randomcolor}{RGB}
   {
    \pdfuniformdeviate 256,
    \pdfuniformdeviate 256,
    \pdfuniformdeviate 256
   }%
}

\newcommand{\randomgray}{%
  \newcount\gray\relax
  \gray=\pdfuniformdeviate 256\relax
  \definecolor{randomgray}{RGB}
   {
    \the\gray,
    \the\gray,
    \the\gray
   }%
}

\newcommand{\numbersquare}[3]{
  \draw #1 rectangle ++(#2, #2) node [midway, black] {#3};
}

\newcommand{\onesquare}[2]{ \numbersquare{#1}{1}{#2} }

%%%%%%%%%%%%%%%%%%%%%%%%%%%%%%%%%%%%%%%%%%%%%%%%%%%%%%%%%%%%%%%%%%%%%%%%%%
% 	NEW ENVIRONMENTS
%%%%%%%%%%%%%%%%%%%%%%%%%%%%%%%%%%%%%%%%%%%%%%%%%%%%%%%%%%%%%%%%%%%%%%%%%%

% http://tex.stackexchange.com/questions/78462/labelling-ax-b-under-an-actual-matrix
\newenvironment{sbmatrix}[1]
 {\def\mysubscript{#1}\mathop\bgroup\begin{bmatrix}}
 {\end{bmatrix}\egroup_{\textstyle\mathstrut\mysubscript}}

\newenvironment{spmatrix}[1]
{\def\mysubscript{#1}\mathop\bgroup\begin{pmatrix}}
{\end{pmatrix}\egroup_{\textstyle\mathstrut\mysubscript}}

\newenvironment{slide}[1]
{
  \centering
	\section{#1}
	\begin{frame}
		\frametitle{#1}
}
{
	\end{frame}
}
